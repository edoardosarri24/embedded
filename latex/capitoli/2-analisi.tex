\chapter{Analisi}
In questo capitolo analiziamo la struttura del progetto, partendo dai suoi componenti e definendo la loro relazione.

%%%%%%%%%%%%%%%%%%%%%%%%%%%%%%%%%%%%%%%%%%%%%%%%%%%%%%%%%%%%%%%%
\section{Componenti}

\subsection{Task}
Un task è definito da:
\begin{itemize}
    \item Pattern di rilascio, cioè una distribuzione del tempo tra due arrivi consecutivi; rappresenta il periodo del task. Ci semplifica la vita rappresnetare la distribuzione con il suo Samplar (vedi Sirio); aggiungiamo a questi il DeterministicSampler parametrizzato dal tempo di rilascio.
    \item Numero e tipo di chunk.
    \item Deadline. Forse conviene considerare la deadline relativa; a lezione abbiamo visto questa soluzione per implementare i vari resource access protocol.
    \item Priorità nominale.
    \item Priorità dinamica.
\end{itemize}

Non ci interessa definire un activation time perché vogliamo considerare il caso pessimo: l'activation time sarà l'istante inziale per tutti i task.

\myskip

Un chunk, cioè una computazione atomica del task, è definito da: una distribuzione del tempo di esecuzione; una eventuale richiesta di risorse da usare in mutua esclusione (da acquisire prima dell'esecuzione e rilascaire subito dopo).

\subsection{Taskset}
È un insieme di task. È l'oggetto principale gestito dallo scheduler.

\subsection{Risorse}
Sono le risorse da utilizzare in mutua esclusione. Ongi risorsa è gestita da un semaforo binario.

\subsection{CPU}
È l'unità di elaborazione. Supponiamo essere unica.

\subsection{Scheduler}
È il componente che assegna un task al processore. Per il momento implementiamo solo Rate Monotonic (RM) e Earliest Deadline First (EDF).

\subsection{Protocollo di accesso alle risorse}
È il meccanismo che garantisce la mutua esclusione di una risorsa. Per il momento implementiamo solo Priority Ceiling Protocol (PCP). Forse poi anche Priority Inheritance Protocol (PIP), ma vediamo.

%%%%%%%%%%%%%%%%%%%%%%%%%%%%%%%%%%%%%%%%%%%%%%%%%%%%%%%%%%%%%%%%
\section{Class diagram}